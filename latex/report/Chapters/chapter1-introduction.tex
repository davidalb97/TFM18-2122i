\chapter{Introduction}
\label{cha:introduction}

\todo[inline]{Electric vehicles are the future}
On today's day and age, the global concern on climate
change has been a major focus on recent international agreements,
such as the Paris Agreement \cite{parisAgreement},
incentivating many car manufacturers to introduce
\gls{EVs} as the eco-friendly
solution for sustainable transport for the future.

%\todo[inline]{EV autonomy}

As \gls{EVs} have been increasing in popularity in
recent years \todo{citar}, car manufacturers have
increased competitiveness on vehicle's performance,
a decisive factor for consumers \cite{EGBUE2012717}.

%\todo[inline]{ERange estimation}

A vehicle's autonomy alson knwon as \gls{eRange},
can be estimated through many driving data parameters,
such as vehicle design, driver's behavior, wheather,
road inclination and gls{SOC} estimation.
The \gls{eRange} accuracy allows consumers to rely
on its vehicle for longer travel time and efficient
charging plans \todo{citar}, \gls{eRange} estimation
however, is a complex problem whitch has prompted
previous studies in the past to provide a solution 
\cite{classicEVX, predictionOfeRange}.

%\todo[inline]{Prior work}
Prior work \cite{classicEVX} on \gls{eRange}
estimation demonstrated that using an
history-based algorithm on am adaptive model
provides a more reliable \gls{eRange} prediction
than a basic \gls{SOC} - manufacturer data relation,
this is mainly due to taking into account the
vehicle's driving history.

\todo[inline]{TODO: Mensionar IA na atualidade}
%\todo[inline]{What will be done}

This project approaches the \gls{eRange} estimation
problem with the use of \gls{machineLearning} based
model to increase its accuracy.

\todo[inline]{TODO: Falar da project structure}