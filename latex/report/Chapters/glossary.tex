\chapter*{Glossary}

\newglossaryentry{eRange}{
    name={eRange},
    description={
        Electric range, the maximum driving range of an electric 
        vehicle using only power from its on-board battery 
        pack to traverse a given driving cycle
    }
}

\newglossaryentry{AI}{
    name={Artificial Intelligence},
    description={code that improoves and learns by itself}
}

\newglossaryentry{machineLearning}{
    name={Machine Learning},
    description={A branch of AI focused on learning from data}
}

\newglossaryentry{dataset}{
    name={Dataset},
    description={A structure containing data for a model}
}

\newglossaryentry{timeSeries}{
    name={Time series},
    description={series of data points indexed by time}
}

\newglossaryentry{python}{
    name={Python},
    description={A hight level programming language}
}

\newglossaryentry{bigData}{
    name={Big Data},
    description={
        Big data is a field that handles with large datasets
        that are too big and complex for traditional data processing
    }
}

\newglossaryentry{neuralNetworks}{
    name={Neural Networks},
    description={
        A collection of connected synapse nodes,
        simulating a biological brain
    }
}
%\newglossaryentry{sample}{ % the label
%    name={sample},                  % the term
%    description={a sample entry  % a brief description
%    }
%}
%
%\newglossaryentry{U}{
%    name={universal set},
%    description={the set of all things
%    },
%    symbol={\ensuremath{\mathcal{U}}
%    }  % the associated symbol
%}
%
%\newglossaryentry{card}{
%    name=cardinality,
%    description={the number of objects within a set
%    },
%    symbol={\ensuremath{|\mathcal{S}|}
%    }
%}
%
%\newglossaryentry{matrix}{
%    name=matrix,
%    description={a rectangular table of elements
%    },
%    plural=matrices
%}