%! suppress = MissingLabel
\chapter{State of Art}
\label{cha:stateOfArt}

\section{EV Range prediction} 

In today's day and age, \gls{EV}s have spired
multiple studies, such as statistical
measurement of charging \citep{EVScout2},
regenerative braking \citep{regenerativeBraking},
charging topologies \citep{batteryChargerTopologies} 
and \gls{eRange} prediction \citep{predictionOfeRange}.

The \gls{eRange} prediction is an important 
\gls{EV} feature to present to consumers as
it reduces driver's anxiety while driving,
This problem has been previously studied before,
prompting multiple ways to tackle the problem.

When proving a solution to the \gls{eRange}
prediction problem, valid \gls{EV} driving data 
in the form of a \gls{dataset} is required to test
the proposed model and compare it to existing
alternatives, making it indispensable in 
determining the effectiveness of the chosen 
solution.

To this end, existing \gls{dataset} solutions
such as the \textit{VED} \gls{dataset} \citep{vedDataset}
that although providing sufficient \gls{EV} driving
data for estimation, there are only three dinstinct 
\gls{EVs} present on the \gls{dataset}, all from the 
same model 2013 \textit{Nissan leaf}.

\todo[inline]{Mencionar datasets inválidos}

Another \gls{dataset} solution is the use of 
a \gls{python} tool \textit{Emobpy} \citep{emobpy}
that generates multiple trips for charging and driving
studies on \gls{EVs}, providing a \gls{dataset}
based on empirical mobility statistics and 
customizable assumptions.

%Defenição do problema

Existing work has demonstrated the use for 
this feature on \gls{EV}s, showing the need
different types of accuracy on \gls{eRange}
estimation depending on the \gls{SOC} state \citep{eRange},
this approach minimizes the performance impact
of minimum cost route searching from high accuracy
\gls{eRange} prediction.

Other studies have focused on delivering 
higher \gls{eRange} estimation accuracy,
making use of more complex models. 
The use of an adaptive history based model 
approach \citep{classicEVX} that relies on
past information about vehicle's instant 
consumption energy, to determine an
adaptive average energy consumption.

\section{Machine Learning and EV Range prediction}

The use of machine learning for a multitude
of cases \citep{machineLearningCaseStudy} in fields such as 
big data \citep{machineLearningBigData, machineLearningBigData2}
and data mining \citep{businessDataMining} has 
proven its robustness on solving complex problems.

As a result, some approaches for the \gls{eRange}
problem have already applied machine learning 
for solving it, most commonly using \gls{neuralNetworks}, 
linear regression \citep{eRangeMachineLearningNeuralnetworkMLR},
\gls{RTs} and \gls{SOM} \citep{eRangeMachineLearningGHSOM}.

\gls{DTs}, \gls{RF}, and \gls{KNN}, have already been 
used in ensemble stacked generalization (ESG) approach 
\citep{eRangeMachineLearningEnsemble} proving its 
effectiveness.

Approaches using \gls{RTs} with gradient boosting 
provide better predictive performance from
ensemble methods when using multiple learning algorithms
such as of \gls{XGBoost} and \gls{LightGBM}
\citep{machineLearningERangeGradientBoostRts}.

Studies in hybrid models of \gls{RTs} and \gls{SOM} 
have improved upon previous solutions by
keeping meaningful knowledge extraction on bushy trees
\citep{machineLearningERangeSOMandRts}.

