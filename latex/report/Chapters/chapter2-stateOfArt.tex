%! suppress = MissingLabel
\chapter{State of Art}
\label{cha:stateOfArt}

\section{EV Range prediction} 
\label{sec:stateOfArtER}

Nowadays, \glspl{EV} have motivated multiple studies
concerning related problems in this field,
such as statistical measurement of charging \citep{EVScout2},
regenerative braking \citep{regenerativeBraking},
charging topologies \citep{batteryChargerTopologies} 
and \gls{eRange} prediction \citep{predictionOfeRange}.

The \gls{eRange} prediction is an important 
\gls{EV} feature to present to consumers as
it reduces driver's anxiety while driving.
This has been previously studied before,
prompting multiple ways to tackle the problem.

When proving a solution to the \gls{eRange}
prediction problem, valid \gls{EV} driving data 
in the form of a \gls{dataset} is required to 
learn and test the proposed model,
and compare it to existing alternatives, 
making it indispensable in determining the
effectiveness of the chosen solution.

To this end, existing \glspl{dataset}
such as the \textit{VED} \gls{dataset} \citep{vedDataset}
that although providing sufficient \gls{EV} driving
data for estimation, there are only three dinstinct 
\glspl{EV} present on the \gls{dataset}, all from the 
same model 2013 \textit{Nissan leaf}.

%\todo[inline]{Mencionar datasets inválidos}

Another \gls{dataset} solution is the use of 
a \gls{python} tool \textit{Emobpy} \citep{emobpy}
that generates multiple trips for charging and driving
studies on \glspl{EV}, providing a \gls{dataset}
based on empirical mobility statistics and 
customizable assumptions.

%Defenição do problema

Existing work has demonstrated the use of
\gls{eRange} estimation on \glspl{EV}, 
showing the need for different types 
of accuracy on \gls{eRange}
estimation depending on the \gls{SOC} state,
the proposed approach by \citep{eRange}
minimizes the performance impact
of minimum cost route searching from high accuracy
\gls{eRange} prediction.

Other studies have focused on delivering 
higher \gls{eRange} estimation accuracy,
making use of more complex models. 
The use of an adaptive history based model 
approach was proposed by \citep{classicEVX},
which relies on past information about
vehicle's instant consumption energy, to determine an
adaptive average energy consumption.

\section{Machine Learning and EV Range prediction}
\label{sec:stateOfArtML}

The use of machine learning for a multitude
of cases \citep{machineLearningCaseStudy} in fields such as 
big data \citep{machineLearningBigData, machineLearningBigData2}
and data mining \citep{businessDataMining} has 
proven its robustness on solving complex problems.


As a result, some approaches for the \gls{eRange}
problem have already applied machine learning,
most notably supervised learning to achieve the estimation.
%Supervised Learning:
\Glspl{DT}, \gls{RF}, and \gls{KNN}, have been 
used in \gls{ESG} approach 
\citep{eRangeMachineLearningEnsemble} proving its 
effectiveness in yielding more acceptable values
for proposed evaluation metrics.
Recent models using \glspl{GBRT} have
combined \gls{XGBoost} and \gls{LightGBM} 
to provide better predictive performance
from these \gls{ensemble} methods 
\citep{machineLearningERangeGradientBoostRts}.

Approaches using unsupervised clustering 
of \glspl{SOM} have been used for clustering \gls{bigData} 
into driving patterns, prior to range estimation 
\citep{eRangeMachineLearningGHSOM}.
The hybrid version of \gls{SOM} with \glspl{RT} 
has taken advantage of \gls{SOM}'s neurons storage 
feature of nearing related neighbor information
being kept closely together, performing \glspl{RT}
locally, avoiding \glspl{bushyTree} and improving
upon previous solutions by keeping 
\gls{meaningfulKnowledgeExtraction} on \glspl{bushyTree}
\citep{machineLearningERangeSOMandRts}.

%Reinforcement Learning + Supervised Learning
\Gls{reinforcementlearning} in the form of \glspl{neuralNetwork}
has also been used for external energies 
disturbances on the speed profile of a driving profiles 
so that it could then be combined with \gls{MLR} 
for the estimation \citep{eRangeMachineLearningNeuralnetworkMLR}.

Although more complex than previous solutions, 
the use of \gls{machineLearning} for the 
\gls{eRange} estimation problem had reduced 
the prediction error, and thus further justifying
its usage in this project.  


