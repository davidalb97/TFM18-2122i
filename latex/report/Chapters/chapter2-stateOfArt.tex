%! suppress = MissingLabel
\chapter{State of Art}
\label{cha:stateOfArt}

\section{EV Range prediction}

%Defenição do problema
Previous studies have already tackled 
various \gls{EV} related topics, for example, 
the statistical measurement of charging \gls{EV}s
\citep{EVScout2}.

\todo[inline]{What is EV range prediction - the problem}
In today's day and age, the \gls{EV} range prediction 
problem has been previously studied before,
prompting multiple ways to tackle the problem \citep{predictionOfeRange}.

Existing work has demonstrated the use for 
this feature on \gls{EV}s, showing the need
different types of accuracy on \gls{eRange}
estimation depending on the \gls{SOC} state \citep{eRange},
this approach minimizes the performance impact
of minimum cost route searching from high accuracy
\gls{eRange} prediction.

Other studies have focused on delivering 
higher \gls{eRange} estimation accuracy,
making use of more complex models. 
The use of an adaptive history based model 
approach \citep{classicEVX} that relies on
(... \todo{descrever}).

Another approach has used machine learning in a hybrid model
of \gls{SOM} integrating \gls{RTs} \citep{hybridMachineLearningERange}
 (... \todo{descrever}).

The \gls{dataset} used to obtain the results
from past studies have an important role
in determining the effectiveness of the chosen 
solution.

To this end, existing \gls{dataset} solutions
already exist such as the VED \gls{dataset} \citep{vedDataset}
(althouth containing only three electric vehicles
\- model 2013 \textit{Nissan leaf}).

Another adopted \gls{dataset} solution is the use of 
a \gls{python} tool \textit{Emobpy} \citep{emobpy}
that generates multiple trips for charging and driving
studies on \gls{EVs}  (... \todo{descrever}).

\section{Machine Learning and EV Range prediction}

\todo[inline]{Machine learning usage on complex problems}

\todo[inline]{Machine learning usage EV world}

\todo[inline]{Machine learning algorithms for }
\todo{referir as técnicas mas especificar noutros capítulos}
\todo{resumo das técnicas usadas no fim do state of the art}
\todo[inline]{classicEVX history based comparisson}


